\documentclass[10pt,a4paper,twosides,titlepage,%
			   headinclude,footinclude,BCOR5mm,%
			   numbers=noenddot,cleardoublepage=empty,%
			   ]{scrreprt}

\usepackage[pdftex]{graphicx}
\usepackage[utf8x]{inputenc}
\usepackage[italian]{babel}

\usepackage{subfig}
\usepackage[subfig,beramono,%
			eulermath,pdfspacing]{classicthesis}

\usepackage{xcolor}
\definecolor{commenti}{rgb}{0.13,0.55,0.13}
\definecolor{stringhe}{rgb}{0.63,0.125,0.94}
\usepackage{listings}
% Carica le impostazioni per importare script matlab
\lstloadlanguages{Matlab}
% Impostazioni per rappresentazioni script
\lstset{% general command to set parameter(s)
basicstyle =\ttfamily , % print whole listing small
keywordstyle = \color{blue},% blue keywords
identifierstyle =\bf, % nothing happens
frame=shadowbox,
commentstyle = \color{commenti}, % comments
stringstyle = \ttfamily \color{stringhe}, % typewriter type for strings
showstringspaces = false, % no special string spaces
%emph = {for, if, then, else, end},
%emphstyle = \color{blue},
firstnumber = 1, % numero della prima linea
numbers = left, %  show number_line
numbersep = 5pt,
language = {Matlab}, % per riconoscere la sintassi matlab
extendedchars = true, % per abilitare caratteri particolari
breaklines = true, % per mandare a capo le righe troppo lunghe
breakautoindent = true, % indenta le righe spezzate
breakindent = 30pt, % indenta le righe di 30pt
}


\usepackage{arsclassica}
\usepackage{amsmath}
\usepackage{amssymb}
\usepackage[retainorgcmds]{IEEEtrantools}
\usepackage[wide,raggedright]{sidecap}
\usepackage[left=1.5in,right=1in,top=1in,bottom=1in,twoside]{geometry}
