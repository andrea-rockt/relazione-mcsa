In questa sottosezione ci occuperemo di descrivere nel dettaglio le equazioni
facenti capo a ciascun blocco funzionale precedentemente descritto:
%###
\paragraph{ Generatore dei riferimenti - Guidance:}
La funzione di questo blocco è quella di ricostruire il quaternione di
riferimento associato al riferimento LORF (Espresso nel sistema di riferimento
J2000) a partire dalle misure ricavate dall'unità GPS ($y_r$ posizione, $y_v$
velocità), richichiamando la definizione del LORF segue che:
\begin{equation}
	\mathfrak{R_0} = \{C,\bar{o}_1,\bar{o}_2,\bar{o}_3\}
\end{equation}

\begin{equation}
\bar{o}_1 = \frac{\vec{v}}{|\vec{v}|} \hspace{20pt}%
\bar{o}_2 = \frac{\vec{h}=\vec{r} \times \vec{v}}{|\vec{h}|} \hspace{20pt}%
\bar{o}_3 = \frac{\vec{e}=\bar{o}_1 \times \bar{o}_2}{|\vec{e}|}
\end{equation}

\begin{equation}
	\components{y_r}(t) = \components{r}(t) + \delta \components{r}(t)
\end{equation}

\begin{equation}
	\components{y_v}(t) = \components{v}(t) + \delta \components{v}(t)
\end{equation}

La matrice d'assetto LORF può essere ricavata come:
\begin{equation}
\underline{R_{0}^{i}}=
\begin{bmatrix}
	\frac{\components{y_v}}{|\components{y_v}|} &
	\frac{\components{y_h}=\components{y_r}	\times \components{y_v}}{|\components{y_h}|} &
	\frac{\components{y_e}=\components{y_v} \times (\components{y_r}\times\components{y_v})}{|\components{y_v}|}(t))
\end{bmatrix}
\end{equation}

La misura proventiente dal GPS é però affetta da errore per tanto definiamo la
matrice tenendo conto di una rotazione infinitesima definita
tramite gli angoli di eulero $\delta\components{\theta_0}$:
\begin{equation}
R_{0}^{i}=\underline{R_{0}^{i}}(I+\delta R_{0}^{i})
\end{equation}
\begin{equation}
\delta R_{0}^{i} = \delta\boldsymbol{\theta_0} \times
\end{equation}
Un approssimazione degli angoli di eulero associati all'errore si ottiene
tramite la seguente relazione:

\begin{equation}
	\delta \boldsymbol{\theta_0}= \frac{1}{r}
	\begin{bmatrix}
	-\delta\vec{r}\cdot\vec{o}_2\\
	\delta\vec{r}\cdot\vec{o}_1-\frac{-\delta\vec{v}\cdot\vec{o}_3}{\omega_0}\\
	\frac{-\delta\vec{v}\cdot\vec{o}_2}{\omega_0}\\
	\end{bmatrix}
,\hspace{10pt} v\approx\omega_0 r 
\end{equation}

Calcoliamo adesso la varianza dell'errore di rotazione maggiorandola con una
quantità $\sigma^2_{0,max}$:
\begin{equation}
\xi[\delta\boldsymbol{\theta_0}
\delta\boldsymbol{\theta_0}^T]\le I \sigma^2_{0,max}
\end{equation}

Allocando la stessa varianza sia alla componente di posizione che di velocità
otteniamo che:

\begin{equation}
\sigma_r < \frac{(R_E + h)\cdot\sigma_{0,max}}{\sqrt{2}}
\end{equation}

\begin{equation}
\sigma_v < \frac{\omega_0\cdot r\cdot \sigma_{0,max}}{\sqrt{2}}
\end{equation}

tramite queste relazioni è possibile scegliere il sensore in grado di garantire
un errore angolare accettabile a partire dalle sue specifiche (varianza in
posizione e velocità)
%###