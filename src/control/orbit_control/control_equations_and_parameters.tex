Come già introdotto nel paragrafo precedente il controllo orbitale è basato
sull' embedded model descritto dalle seguenti equazioni:
\begin{IEEEeqnarray}{rCl}
	x(i+1)&=&(1-\beta)x(i) + \beta(d(i)+w_0(i)+u(i))\nonumber\\
	y(i)&=&x(i)+e(i) = y_m(i)+ e(i)
\end{IEEEeqnarray}
e sull'equazione di comando:
\begin{equation}
	u(i)= -X_{d1}(i)
\end{equation}
In questa topologia risulta di grande importanza la descrizione dei disturbi,
costituiti da una parte prevedibile ($X_{d1}(i)$) e da un rumore bianco
rappresentante la parte non prevedibile ($w_0(i)$)
\begin{equation}
	d(i)=X_{d1}(i)+w_0(i)
\end{equation}
Modellare il disturbo su base statistica permette al controllo di essere
autonomo (non si richiedono update di parametri da terra) ma implica che il
modello sia in qualche modo influenzato dalla realtà.
Nel caso preso in esame il modello statistico prevede un rumore bianco
rappresentante la caratteristica dei propulsori e una componente del secondo
ordine rappresentante le forze aerodinamiche.
%###
\paragraph{Forze aerodinamiche:} A partire da risultati simulati si è evinto
come la densità spettrale di potenza decresca di $-40db/dec$ per tanto
rappresentante una caratteristica del secondo ordine, si è quindi modellata tale
caratteristica su base statistica tramite una doppia deriva aleatoria in grando
quindi di descrivere correttamente l'ampia classe di segnali associati alle
forze aerodinamiche.
%###

Modellare un disturbo su base statistica implica un aggiornamento del modello
sulla base dell'errore di modello, ciò avviene estrapolando le informazioni
contenute nell'errore di modello ($e(i)=y(i)-y_m(i)$) attraverso il {\bf Noise
Estimator} costituito dal vettore di guadagni:

\begin{equation}
	L=\begin{bmatrix}
	l_0\\
	l_1\\
	l_2\\
	\end{bmatrix}
\end{equation}

La parte di sistema costituita dall'embedded model e dal noise estimator diventa
quindi un predittore dello stato in catena chiusa e il vettore di guadagni $L$
ci permette di schedulare opportunamente i suoi autovalori permettendoci di
effettuare la predizione a un passo dello stato $x_{d1}$ potendo quindi
utilizzarlo nella legge di controllo prevedendo un opportuno blocco di ritardo
\emph{D}
