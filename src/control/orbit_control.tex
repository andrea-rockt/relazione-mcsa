Blocco funzionale fondamentale dell'architettura di controllo di un satellite è
il controllo orbitale, tale blocco assolve al compito di cancellare le forze
non gravitazionali per mantenere il satellite in caduta libera (Drag-Free),
esistono due principali approcci al controllo drag free:

\begin{description}
\item[Basato su massa di prova:]  Un corpo di massa piccola (solitamente
$<1Kg$) viene sospeso libero di muoversi in una gabbia all'interno del satellite
(la massa di prova è quindi idealmente non soggetta a forze di tipo non
gravitazionale), si fa quindi in modo che il satellite ``insegua'' la massa ed
essendo la massa idealmente in caduta libera il satellite tenderà ad essere in
caduta libera. \'E possibile utilizzare una sola massa di prova.
\item[Basato su accelerometro:] L'intero satellite è reso drag free effettuando
misure dirette delle forze non gravitazionali ed attuandolo per compensarle, i
corpi all'interno del satellite idealmente ``fluttueranno''. \'E possibile
utilizzare un numero arbitrario di accelerometri per ottenere le misure
necessarie a rendere il satellite Drag-Free.
\end{description}

%###
\subsection{Architecture and objectives}
Obiettivo del controllo orbitale drag-free è la cancellazione delle forze non
gravitazionali agenti sul satellite per rendere il moto del centro di massa del
satellite dovuto con accettabile approssimazione esclusivamente alle forze
gravitazionali. L'accelerazione del centro di massa del satellite è descritta
dalle seguenti relazioni (tutte le coordinate sono espresse nel riferimento
corpo):

\begin{equation}
\components{\dot{v}}(t) = \components{g}(\components{r}(t))+
\frac{R_b(\quat{q})(-\components{F_d}(t)+\components{F_t}(t))}{m},
\components{v(0)=v_	0}
\end{equation}
dove $\components{F_d}$ sono le forze non gravitazionali (principalmente
aerodinamiche in orbita bassa) e $\components{F_t}$ sono le forze di comando
dovute ai propulsori.

Il controllo drag-free è soddisfatto quando la seguente relazione è soddisfatta:

\begin{equation}
\components{a}(t) = \frac{(- \components{F_d}(t) + \components{F_t}(t))}{m}=0
\end{equation}
ovvero l'accelerazione residua dovuta a componenti non gravitazionali
($\components{a}$) è nulla.

L'architettura prescelta per la compensazione delle forze suddette è basata 
sull' embedded model i cui concetti fondamentali sono:
\begin{description}
\item[Dinamica Controllabile:] Descrive la dinamica tra comando e misura, deve
catturare le dinamiche ad alta frequenza il più vicino possibile alla più alta
componente in frequenza dei disturbi da cancellare.
\item[Dinamica del Disturbo:] Dinamica rappresentante il disturbo e modellata su
base statistica.
\end{description} 

\begin{figure}
\includegraphics[width=\textwidth]{control/orbit_control/images/block-diagram.pdf}
\caption{Architettura del controllo}
\end{figure}

\subsection{Control equations and parameters}
Come già introdotto nel paragrafo precedente il controllo orbitale è basato
sull' embedded model descritto dalle seguenti equazioni:
\begin{IEEEeqnarray}{rCl}
	x(i+1)&=&(1-\beta)x(i) + \beta(d(i)+w_0(i)+u(i))\nonumber\\
	y(i)&=&x(i)+e(i) = y_m(i)+ e(i)
\end{IEEEeqnarray}
e sull'equazione di comando:
\begin{equation}
	u(i)= -X_{d1}(i)
\end{equation}
In questa topologia risulta di grande importanza la descrizione dei disturbi,
costituiti da una parte prevedibile ($X_{d1}(i)$) e da un rumore bianco
rappresentante la parte non prevedibile ($w_0(i)$)
\begin{equation}
	d(i)=X_{d1}(i)+w_0(i)
\end{equation}
Modellare il disturbo su base statistica permette al controllo di essere
autonomo (non si richiedono update di parametri da terra) ma implica che il
modello sia in qualche modo influenzato dalla realtà.
Nel caso preso in esame il modello statistico prevede un rumore bianco
rappresentante la caratteristica dei propulsori e una componente del secondo
ordine rappresentante le forze aerodinamiche.
%###
\paragraph{Forze aerodinamiche:} A partire da risultati simulati si è evinto
come la densità spettrale di potenza decresca di $-40db/dec$ per tanto
rappresentante una caratteristica del secondo ordine, si è quindi modellata tale
caratteristica su base statistica tramite una doppia deriva aleatoria in grando
quindi di descrivere correttamente l'ampia classe di segnali associati alle
forze aerodinamiche.
%###

Modellare un disturbo su base statistica implica un aggiornamento del modello
sulla base dell'errore di modello, ciò avviene estrapolando le informazioni
contenute nell'errore di modello ($e(i)=y(i)-y_m(i)$) attraverso il {\bf Noise
Estimator} costituito dal vettore di guadagni:

\begin{equation}
	L=\begin{bmatrix}
	l_0\\
	l_1\\
	l_2\\
	\end{bmatrix}
\end{equation}

La parte di sistema costituita dall'embedded model e dal noise estimator diventa
quindi un predittore dello stato in catena chiusa e il vettore di guadagni $L$
ci permette di schedulare opportunamente i suoi autovalori permettendoci di
effettuare la predizione a un passo dello stato $x_{d1}$ potendo quindi
utilizzarlo nella legge di controllo prevedendo un opportuno blocco di ritardo
\emph{D}
\subsection{Simulated and explained plots}
Le simulazioni sono state ottenute abilitando nel simulatore le componenti
corrispondenti alle seguenti flags:
\begin{lstlisting}[language=matlab,breaklines=true]
GravityTypeFlag=1;		%(1=J2 Gravity Model/=0 Spherical)
GravityGradientTorqueFlag=1;	%1=Gravity Gradient Torque ON/0=OFF 
DragForceDisturbancesFlag=1;	%0=Drag Force Disturbance OFF/1=OFF
DragTorquesDisturbancesFlag=1;	%0=Drag Torque Disturbance OFF/1=ON
DragFreeControlFlag=1;		%0=Drag Free Control OFF/1=ON
AttitudeControlFlag=1;		%0=Attitude Control OFF/1=ON
\end{lstlisting}

\begin{SCfigure}[0.7][ht]
	\includegraphics[width=.6\textwidth]{control/attitude_control/images/EmbeddedModelAttitude.pdf}
	\caption{\emph{Quaternione di assetto} --- Quaternione assetto stimato
	dall'embedded model a partire dallo star tracker}
	\label{fig:drag-acceleration}
\end{SCfigure}

\begin{SCfigure}[0.7][ht]
	\includegraphics[width=.6\textwidth]{control/attitude_control/images/ReferenceQuaternion.pdf}
	\caption{\emph{Quaternione di riferimento} --- Quaternione rappresentante il
	riferimento LORF nel sistema di riferimento inerziale, deve essere inseguito
	dal controllo}
	\label{fig:drag-acceleration}
\end{SCfigure}

\begin{SCfigure}[0.7][ht]
	\includegraphics[width=.6\textwidth]{control/attitude_control/images/TrackingError.pdf}
	\caption{\emph{Errore di inseguimento} --- L'errore si mantiene nei requisiti
	di precisione richiesti}
	\label{fig:drag-acceleration}
\end{SCfigure}
%###