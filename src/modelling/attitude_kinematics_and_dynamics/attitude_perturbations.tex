L'interazione del satellite con l'ambiente circostante genera coppie che ne
disturbano l'assetto. Le principali perturbazioni sono:
\begin{description}
\item[Gradiente di gravità] generato dalla non uniformità del campo
gravitazionale terrestre, causa coppie disturbanti soprattutto a orbite basse
\item[Radiazioni elettromagnetiche] provenienti dal Sole e dagli altri pianeti,
queste onde urtano il satellite alterandone l'assetto
\item[Forze aerodinamiche] particelle esterne, dovute all'atmosfera terrestre,
colpiscono la superficie del satellite causando una pressione e quindi forze e
coppie che disturbano sia l'orbita che l'assetto
\item[Campo magnetico terrestre] interagisce con il dipolo elettrico del
satellite generato dalla strumentazione elettronica
\end{description}

Le coppie generate dall'interazione col gradiente gravitazionale terrestre sono
dovute alla non uniformità della distribuzione di massa della Terra e dalla
sua non sfericità, infatti l'accelerazione gravitazionale terrestre risulta
essere in funzione della distanza $\vec{r}$ dal centro della Terra
\begin{equation}
|\vec{g}(\vec{r}_c)|\approx9.81 [m/s^2], \ \vec{r}_c=R_E
\end{equation}
dove $R_E$ è la misura del raggio equatoriale della Terra.
Indicando con $\vec{s}$ la posizione di un generico punto del satellite rispetto
al centro di massa di quest'ultimo, possiamo calcolare la coppia di gravità
\begin{equation}
{\bf M}_g(t)=\int_V\vec{s}\times\vec{g}(\vec{r})dm
\end{equation}
sviluppando il vettore accelerazione gravitazionale, si ottiene
\begin{equation}
{\bf M}_g(t)=\int_V\vec{s}\times U_b({\bf r}_c){\bf s}dm
\label{eq:coppia_gravita}
\end{equation}
dove $U_b({\bf r}_c)$ è la matrice del gradiente di gravità nelle coordinate
corpo. Se gli assi corpo coincidono con gli assi principali di inerza,
l'equazione \ref{eq:coppia_gravita} diventa
\begin{equation}
{\bf M}_g(t)=\int_V 
\begin{bmatrix}
0 & -s_3 & s_2 \\
s_3 & 0 & -s_1 \\
-s_2 & s_1 & 0
\end{bmatrix}
[U_b]\begin{bmatrix}
s_1 \\ s_2 \\ s_3
\end{bmatrix} dm,
\end{equation}
con
\[ U_b=
\begin{bmatrix}
(J_3-J_2)U_{23} \\ (J_1-J_3)U_{13} \\ (J_2-J_1)U_{12}
\end{bmatrix}
\]
da cui si deduce che in caso di simmetria sferica ${\bf M}_g=0$

Per semplificare i calcoli ipotizziamo una gravità sferica, sotto questa ipotesi
possiamo scrivere
\begin{equation}
{\bf g}({\bf r}_c)=-\left(\frac{\mu_E}{r_c^3}\right){\bf r}_c
\end{equation}
dove $\mu_E=Gm_E$, adesso ricalcoliamo il tensore di gravità, che descrive come
varia nello spazio l’accelerazione di gravità dovuta ad un gravità sferica.
\begin{equation}
{\bf M}_g(t)=\int_V\vec{s}\times U_b({\bf r}_c){\bf s} \ dm=\frac{3\mu_E}{r_c^5}
\begin{bmatrix}
(J_3-J_2)y_cz_c \\ (J_1-J-3)x_cz_c \\ (J_2-J_1)x_cy_c
\end{bmatrix}
\end{equation}
Il che dimostra come in caso di simmetria cilindrica, la terza componente non
appare. Inoltre la coppia del gradiente gravità risulta essere ortogonale alla
verticale locale $\bar{r}_c$ e diminuisce col cubo della distanza dal satellite
al centro di massa del pianeta.

Adesso studiamo il disturbo causato dalle coppie aerodinamiche, assumendo che
l'impatto delle particelle dell'atmosfera sulla superficie del satellite sia
anaelastico e senza riflessione, quindi l'energia è totalmente assorbita.
Indichiamo con $\vec{v}_r$ la velocità relativa tra il centro di massa del
satellite e l'atmosfera circostante e con $\bar{e}_v$ il suo versore. Sia l'area
di impatto pari a $\cos\alpha dA$, dove $\cos\alpha=\bar{n}\times\bar{e}_v\geq
0$ e $\alpha$ è l'angolo di incidenza. Definiamo $C_D$ il coefficiente di drag
(attrito).
Sotto queste ipotesi la velocità relativa $\vec{v}_r$ risulta essere
\begin{equation}
\vec{v}_r=\vec{v}-\vec{\omega}_p\times\vec{r}-\vec{w}
\label{eq:vel_relativa}
\end{equation}
dove $\vec{w}$ è la velocità del vento, $\vec{\omega}_p$ è la rotazione angolare
del pianeta e $\vec{r}$ e $\vec{v}$ sono la posizione e la velocità del centro
di massa del satellite.
Nelle coordinate del sistema di riferimento inerziale centrato nel centro di
massa del pianeta, l'equazione \ref{eq:vel_relativa} diventa
\begin{equation}
\begin{bmatrix}
v_{r1} \\ v_{r2} \\ v_{r3}
\end{bmatrix} (t) = 
\begin{bmatrix}
v_1 \\ v_2 \\ v_3
\end{bmatrix}
(t) + \omega_p
\begin{bmatrix}
r_2 \\ -r_1 \\ 0
\end{bmatrix}(t) -
\begin{bmatrix}
w_1 \\ w_2 \\ w_3
\end{bmatrix}(t)
\end{equation}
La forza di impatto è proporzionale alla pressione atmosferica $\rho$
\begin{equation}
P_a=\frac{1}{2}\rho|v_r|^2
\end{equation}
Inserendo l'angolo di incidenza e il coefficiente di attrito, otteniamo la forza
di drag risultante
\begin{equation}
d\vec{F}_D=\frac{1}{2}\rho|v_r|^2C_D\cos\alpha\bar{e}_v
\end{equation}
Infine, sommando il contributo delle n superfici e ricordando che $a_k$
rappresenta il punto di applicazione,o centro di pressione, otteniamo la coppia
totale dovuta al drag
\begin{equation}
\vec{M}_D=\sum_{k=1}^n\vec{a}_k\times\vec{F}_{Dk}
\end{equation}
