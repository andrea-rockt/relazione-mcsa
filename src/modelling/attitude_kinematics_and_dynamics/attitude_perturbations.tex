L'interazione del satellite con l'ambiente circostante genera coppie che ne
disturbano l'assetto. Le principali perturbazioni sono:
\begin{description}
\item[Gradiente di gravità] generato dalla non uniformità del campo
gravitazionale terrestre, causa coppie disturbanti soprattutto a orbite basse
\item[Radiazioni elettromagnetiche] provenienti dal Sole e dagli altri pianeti,
queste onde urtano il satellite alterandone l'assetto
\item[Forze aerodinamiche] particelle esterne, dovute all'atmosfera terrestre,
colpiscono la superficie del satellite causando una pressione e quindi forze e
coppie che disturbano sia l'orbita che l'assetto
\item[Campo magnetico terrestre] interagisce con il dipolo elettrico del
satellite generato dalla strumentazione elettronica
\end{description}

Le coppie generate dall'interazione col gradiente gravitazionale terrestre sono
dovute alla non uniformità della distribuzione di massa della Terra e dalla
sua non sfericità, infatti l'accelerazione gravitazionale terrestre risulta
essere in funzione della distanza $\vec{r}$ dal centro della Terra
\begin{equation}
|\vec{g}(\vec{r}_c)|\approx9.81 [m/s^2], \ \vec{r}_c=R_E
\end{equation}
dove $R_E$ è la misura del raggio equatoriale della Terra.
Indicando con $\vec{s}$ la posizione di un generico punto del satellite rispetto
al centro di massa di quest'ultimo, possiamo calcolare la coppia di gravità
\begin{equation}
{\bf M}_g(t)=\int_V\vec{s}\times\vec{g}(\vec{r})dm
\end{equation}
sviluppando il vettore accelerazione gravitazionale, si ottiene
\begin{equation}
{\bf M}_g(t)=\int_V\vec{s}\times U_b({\bf r}_c){\bf s}dm
\label{eq:coppia_gravita}
\end{equation}
dove $U_b({\bf r}_c)$ è la matrice del gradiente di gravità nelle coordinate
corpo. Se gli assi corpo coincidono con gli assi principali di inerza,
l'equazione \ref{eq:coppia_gravita} diventa
\begin{equation}
{\bf M}_g(t)=\int_V 
\begin{bmatrix}
0 & -s_3 & s_2 \\
s_3 & 0 & -s_1 \\
-s_2 & s_1 & 0
\end{bmatrix}
[U_b]\begin{bmatrix}
s_1 \\ s_2 \\ s_3
\end{bmatrix} dm,
\end{equation}
con
\[ U_b=
\begin{bmatrix}
(J_3-J_2)U_{23} \\ (J_1-J_3)U_{13} \\ (J_2-J_1)U_{12}
\end{bmatrix}
\]
da cui si deduce che in caso di simmetria sferica ${\bf M}_g=0$

Per semplificare i calcoli ipotizziamo una gravità sferica, sotto questa ipotesi
possiamo scrivere
\begin{equation}
{\bf g}({\bf r}_c)=-\left(\frac{\mu_E}{r_c^3}\right){\bf r}_c
\end{equation}
dove $\mu_E=Gm_E$, adesso ricalcoliamo il tensore di gravità, che descrive come
varia nello spazio l’accelerazione di gravità dovuta ad un gravità sferica.
\begin{equation}
{\bf M}_g(t)=\int_V\vec{s}\times U_b({\bf r}_c){\bf s} \ dm=\frac{3\mu_E}{r_c^5}
\begin{bmatrix}
(J_3-J_2)y_cz_c \\ (J_1-J-3)x_cz_c \\ (J_2-J_1)x_cy_c
\end{bmatrix}
\end{equation}
Il che dimostra come in caso di simmetria cilindrica, la terza componente non
appare. Inoltre la coppia del gradiente gravità risulta essere ortogonale alla
verticale locale $\bar{r}_c$ e diminuisce col cubo della distanza dal satellite
al centro di massa del pianeta.
