L'assetto (o orientamento) di un corpo è definito dalla rappresentazione delle
coordinate del sistema riferimento corpo
$\mathfrak{R}_b\{O,\bar{b}_1,\bar{b}_2,\bar{b}_3\}$ in un sistema di riferimento
dell'osservatore, nel nostro caso inerziale
$\mathfrak{R}_i\{O,\bar{i}_1,\bar{i}_2,\bar{i}_3\}$. L'origine di entrambi i
sistemi di riferimento deve coincidere. Dal teorema di Eulero sappiamo che sono
sufficienti tre rotazioni elementari per passare da un riferimento all'altro
\begin{itemize}
  \item $R_1(\theta_1)=X(\phi)$
  \item $R_2(\theta_2)=Y(\theta)$
  \item $R_3(\theta_3)=Z(\psi)$
\end{itemize}
dove i tre angoli rappresentano gli angoli di Eulero. Il prodotto tra matrici
non è commutativo, quindi è necessario scegliere con accuratezza l'ordine delle
rotazioni. Esistono due convenzioni
\begin{itemize}
  \item Proprie di Eulero, utilizza due assi uguali: Z-X-Z
  \item Di Tayt-Brian: Z-Y-X e X-Y-Z
\end{itemize}
Le rotazioni di Eulero si possono rappresentare mediante una singola rotazione
$\phi$ (detta rotazione principale) attorno ad un opportuno asse $\bar{e}$,
indicando la trasformazione con $\mathfrak{R}_i^b\{\phi,\bar{e}\}$.

Per quanto concerne la simulazione, è stato scelto di rappresentare l'assetto
del satellite mediante i quaternioni.
Il quaternione è l'estensione su quattro dimensioni dei parametri di Eulero
$(\phi,\bar{e})$
\begin{equation}
\mathit{q}=\begin{bmatrix}
\mathit{q}_0 \\ \mathit{\bf q}
\end{bmatrix} =\begin{bmatrix}
\cos{\phi/2} \\ {\bf e}\sin{\phi/2}
\end{bmatrix}
\end{equation}
La composizione dei quaternioni avviene come quella degli angoli di Eulero,
infatti se consideriamo n-riferimenti, ad ogni matrice di rotazione corrisponde
un quaternione. Invece nelle rotazioni infinitesime, la composizione è
commutativa e vale la seguente approssimazione
\begin{equation}
\mathit{q}\approx
\begin{bmatrix}
1\\ \phi/2 \ {\bf e}
\end{bmatrix}
\end{equation}
Bisogna tenere conto che è necessaria una normalizzazione, perché si
dimostra che con uno sviluppo in serie arrestato al
primo ordine non vale più la relazione
\begin{equation}
\mathit{q}^T\mathit{q}=1
\end{equation}
che va quindi imposta di nuovo.

Adessio calcoliamo le equazioni di stato della cinematica dei quaternioni.
Consideriamo un quaternione all'istante di tempo $t+\Delta t$ e lo scriviamo in
relazione al suo stato precedente, quindi al tempo $t$
\begin{IEEEeqnarray}{rCl} \nonumber
\mathit{q}(t+\Delta t)&=&\mathit{q}(t)\otimes
\begin{bmatrix}
\cos(\omega(t)\Delta t/2)\\
\sin(\omega(t)\Delta t/2){\bf e}(t)
\end{bmatrix} =\\ &=& \cos(\omega(t)\Delta
t/2)\mathit{q}(t)+\frac{\sin(\omega(t)\Delta
t/2)}{\omega(t)}\mathit{q}(t)\otimes
\begin{bmatrix}
$0$\\
\omega(t){\bf e}(t)
\end{bmatrix} \nonumber
\end{IEEEeqnarray}
Adesso costruiamo il rapporto incrementale
\[
\frac{\mathit{q}(t+\Delta t)-\mathit{q}(t)}{\Delta t}=
\frac{\cos(\omega(t)\Delta t/2)-1}{\Delta t}\mathit{q}(t)+
\frac{1}{2}\frac{\sin(\omega(t)\Delta t/2)}{\omega(t)\Delta
t/2}\mathit{q}(t)\otimes \begin{bmatrix}
0\\ \omega(t){\bf e}(t)
\end{bmatrix}
\]
che per $\Delta t\rightarrow0$ diventa
\begin{equation}
\dot{\mathit{q}}(t)=\frac{1}{2}\mathit{q}(t)\otimes
\begin{bmatrix}
0\\\omega(t){\bf e}(t)
\end{bmatrix}=\frac{1}{2}\mathit{q}(t)\otimes
\begin{bmatrix}
0\\{\boldsymbol {\omega}}(t)
\end{bmatrix}
\end{equation}
Si nota che la derivata di un quaternione è non unitaria, ma di ampiezza $\omega
/2$, ottenuta ruotando il quaternione di $\pi$ attorno all'asse della velocità
angolare $\bar{e}$

Adesso che l'assetto e la variazione di assetto del satellite sono stati
definiti, possiamo indagare sulle cause del moto, quindi andiamo a vedere la
relazione tra l'accelerazione angolare e le forze e i momenti di
forza sia interni che esterni al satellite.
L'equazione di Newton sulla rotazione di un corpo rigido
\begin{equation}
\dot{\vec{H}}=\int_V\vec{r}\times\ddot{\vec{r}}dm=\int_V\vec{r}\times
d\vec{F}=
\vec{M}
\label{eq:newton_eq}
\end{equation} 
ci dice che la velocità di variazione del momento angolare $\dot{\vec{H}}$
attorno a un punto fisso è uguale al momento totale $\vec{M}$ delle forze esterne agenti
sul corpo rigido. Si può dimostrare che la stessa relazione vale per il centro
di massa
\begin{equation}
\dot{\vec{H}}_c=\vec{M}_c
\end{equation}
Consideriamo il momento angolare del centro di massa, esso è definito come
\begin{equation}
\vec{H}_c=J\vec{\omega}
\end{equation}
dove la matrice simmetrica $J$ è chiamata matrice del momento di inerzia del
centro di massa o più semplicemente tensore di inerzia.
Poiché è simmetrica e definita positiva, i suoi tre autovalori sono reali e
positivi, quindi gli autovettori sono reali e ortogonali. Essi, se presi come
terna di assi cartesiani, costituiscono gli assi principali di inerzia. Se il
riferimento corpo del satellite coincide con il riferimento degli assi
principali di inerzia allora $J$ è diagonale
\begin{equation}
J=\begin{bmatrix}
J_1 & 0 & 0\\0 & J_2 & 0\\0 & 0 & J_3
\end{bmatrix}
\end{equation}
Dai valori assunti da $J_1, J_2$ e $J_3$ si può conoscere la distribuzione di
massa del satellite
\begin{itemize}
  \item $J_1=J_2=J_3$: simmetria sferica
  \item $J_1=J_2<J_3$: disco o flat body
  \item $J_1=J_2>J_3$: bacchetta o slim body
\end{itemize}
Calcoliamo la derivata del momento angolare del centro di massa, ipotizzando che
il satellite non ruoti. Poiché le coordinate corpo non sono inerziali, la
derivata di $\dot{\vec{H}}_c$ è la somma della derivata nel sistema di
riferimento corpo e della rotazione del sistema di riferimento corpo
\begin{equation}
\dot{\vec{H}}_c(t)=J\dot{\vec{\omega}}+\vec{\omega}\times J\vec{\omega}
\end{equation}
Riscrivendola in funzione di $\dot{\vec{\omega}}$ e ricordando
l'equazione di Newton \ref{eq:newton_eq}, otteniamo l'equazione fondamentale
della dinamica d'assetto
\begin{equation}
\dot{\vec{\omega}}(t)=-J^{-1}\vec{\omega}(t)\times J\vec{\omega}(t)+J^{-1}M_c(t)
\end{equation}
dove $-J^{-1}\vec{\omega}(t)\times J\vec{\omega}(t)$ rappresenta le
accelerazioni giroscopiche.

Le equazioni di stato del satellite, date come combinazione delle equazioni
della cinematica e della dinamica d'assetto, in forma vettoriale e compatta sono
\begin{equation}
\begin{bmatrix}
\dot{\mathit{\bf q}}\\\dot{\boldsymbol \omega}
\end{bmatrix}(t) =
\begin{bmatrix}
0 & \mathit{Q}(\mathit{\bf q})/2 \\
0 & \Omega({\boldsymbol \omega})=-J^{-1}{\boldsymbol \omega}(t)\times J
\end{bmatrix} 
\begin{bmatrix}
\mathit{\bf q}\\{\boldsymbol \omega}
\end{bmatrix}(t)+
\begin{bmatrix}
0 \\ J^{-1}
\end{bmatrix}{\bf M}(t), 
\end{equation}
\[  {\bf q}(t)^T{\bf q}(t)=1 \]
dove $\mathit{Q}(\mathit{\bf q})$ è definita come
\[ \mathit{Q}(\mathit{\bf q})=
\begin{bmatrix}
-q_1 & -q_2 & -q_3 \\
q_0 & -q_3 & q_2 \\
q_3 & q_0 & -q_1 \\
-q_2 & q_1 & q_0
\end{bmatrix} \]
mentre $\Omega({\boldsymbol \omega})$, nel caso in cui il
riferimento corpo del satellite coincide con il riferimento degli assi
principali di inerzia, si semplifica e risulta essere
\[ \Omega({\boldsymbol \omega})= -J^{-1}{\boldsymbol \omega}(t)\times J =
\begin{bmatrix}
0 & \frac{J_2-J_3}{J1}\omega_3 & 0 \\
\frac{J_3-J_1}{J_2}\omega_3 & 0 & 0 \\
\frac{J_1-J_2}{J_3}\omega_2 & 0 & 0
\end{bmatrix} \]