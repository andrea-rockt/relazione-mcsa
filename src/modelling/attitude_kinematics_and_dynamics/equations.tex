L'assetto (o orientamento) di un corpo è definito dalla rappresentazione delle
coordinate del sistema riferimento corpo
$\mathfrak{R}_b\{O,\bar{b}_1,\bar{b}_2,\bar{b}_3\}$ in un sistema di riferimento
dell'osservatore, nel nostro caso inerziale
$\mathfrak{R}_i\{O,\bar{i}_1,\bar{i}_2,\bar{i}_3\}$. L'origine di entrambi i
sistemi di riferimento deve coincidere. Dal teorema di Eulero sappiamo che sono sufficienti tre rotazioni elementari per passare da un riferimento all'altro
\begin{itemize}
  \item $R_1(\theta_1)=X(\phi)$
  \item $R_2(\theta_2)=Y(\theta)$
  \item $R_3(\theta_3)=Z(\psi)$
\end{itemize}
dove i tre angoli rappresentano gli angoli di Eulero. Il prodotto tra matrici
non è commutativo, quindi è necessario scegliere con accuratezza l'ordine delle
rotazioni. Esistono due convenzioni
\begin{itemize}
  \item Proprie di Eulero, utilizza due assi uguali: Z-X-Z
  \item Di Tayt-Brian: Z-Y-X e X-Y-Z
\end{itemize}
Le rotazioni di Eulero si possono rappresentare mediante una singola rotazione
$\phi$ (detta rotazione principale) attorno ad un opportuno asse $\bar{e}$,
indicando la trasformazione con $\mathfrak{R}_i^b\{\phi,\bar{e}\}$.
Per quanto concerne la simulazione, è stato scelto di rappresentare l'assetto
del satellite mediante i quaternioni.
Il quaternione è l'estensione su quattro dimensioni dei parametri di Eulero
$(\phi,\bar{e})$
\begin{equation}
\mathit{q}=\begin{bmatrix}
\mathit{q}_0 \\ \mathit{\bf q}
\end{bmatrix} =\begin{bmatrix}
\cos{\phi/2} \\ {\bf e}\sin{\phi/2}
\end{bmatrix}
\end{equation}
La composizione dei quaternioni avviene come quella degli angoli di Eulero, se
consideriamo n riferimenti, ad ogni matrice di rotazione corrisponde un
quaternione. Inoltre nelle rotazioni infinitesime, la composizione è commutativa
e vale la seguente approssimazione
\begin{equation}
\mathit{q}\approx
\begin{bmatrix}
1\\ \phi/2 \ {\bf e}
\end{bmatrix}
\end{equation}
prestando attenzione al fatto che con uno sviluppo in serie arrestato al primo
ordine non vale più
\begin{equation}
\mathit{q}^T\mathit{q}=1
\end{equation}
ed è quindi necessaria una normalizzazione.