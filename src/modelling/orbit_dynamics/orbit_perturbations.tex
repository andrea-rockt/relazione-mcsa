Adesso teniamo conto delle perturbazioni che l'orbita del satellite subisce,
andando quindi a calcolare la risposta forzata, considerando la non sfericità
della Terra e la presenza delle forze non gravitazionali. Vi sono diversi tipi
di orbite che possono essere realizzate, nel nostro caso l'orbita sarà
elio-sincrona, quindi i raggi solari andranno a colpire sempre la stessa faccia
del satellite. Poiché la direzione Sole-Terra ruota per tutta la durata
dell'anno solare, il piano orbitale non può rimanere fisso nello spazio, ma la
linea dei nodi deve ruotare, introducendo così il moto di precessione. Per fare
ciò bisogna eguagliare la velocità di rotazione del satellite a quella
dell'orbita terrestre rispetto al Sole
\begin{equation}
\omega_s\approx \frac{2\pi}{365.25\times86400}\approx 0.2 \ \mu rad/s
\end{equation}
Un'orbita di questo tipo è non kepleriana, ma possiamo sfruttare le
perturbazioni esterne per assicurarci il moto di precessione. Si può dimostrare
che grazie alll'appiattimento terrestre, ad una specifica inclinazione del piano
orbitale, dipendente dall'eccentricità e dalla lunghezza del semi-asse maggiore,
il moto di precessione viene perseguito in maniera naturale. Considerando il
nostro caso, ovvero un'orbita ad una bassa altitudine e una piccola eccentricità
e considerando la costante $J_2$, definita come la seconda armonica zonale (essa
rappresenta l'appiattimento ai poli del globo terrestre), per assicurarci il
moto di precessione l'orbita dev'essere di tipo quasi-polare.
