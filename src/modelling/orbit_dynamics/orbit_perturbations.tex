Adesso teniamo conto delle perturbazioni che l'orbita del satellite subisce,
andando quindi a calcolare la risposta forzata, considerando la non sfericità
della Terra e la presenza delle forze non gravitazionali. Vi sono diversi tipi
di orbite che possono essere realizzate, nel nostro caso l'orbita sarà
elio-sincrona, quindi i raggi solari andranno a colpire sempre la stessa faccia
del satellite. Poiché la direzione Sole-Terra ruota per tutta la durata
dell'anno solare, il piano orbitale non può rimanere fisso nello spazio, ma la
linea dei nodi deve ruotare, introducendo così il moto di precessione.
