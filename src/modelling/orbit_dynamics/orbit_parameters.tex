Per conoscere i parametri orbitali dobbiamo prima calcolare la risposta libera
del nostro sistema. Riscriviamo l'equazione $\ref{eq:2body_acc}$ senza
considerare le forze esterne
\begin{equation}
\dot{v}(t)=-\frac{\mu_0}{r^3}{\bf r(t)}, v(0)=v_0
\end{equation}
La risposta libera descrive un'orbita che giace su un piano fisso, il piano
orbitale, definito dai vettori velocità e posizione. Si nota che l'accelerazione
è istante per istante opposta al vettore posizione, quindi se prendiamo in esame
il momento angolare per unità di massa {\bf h(t)}, definito come ${\bf
h(t)}={\bf r(t)}$ x ${\bf v(t)}$, la sua derivata è uguale a zero. Quindi ${\bf
h}=h{\bf e_h}$ è un vettore costante in direzione ${\bf e_h}$ (il polo dell'orbita) e
magnitudine h, che può essere presa come asse del sistema di riferimento.
Mediante il momento angolare possiamo anche definire il vettore eccentricità
\[ {\bf v}(t) \times {\bf h} - \frac{\mu}{r(t)}{\bf r}(t)=\mu{\bf e} \]
