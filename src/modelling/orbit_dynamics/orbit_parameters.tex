Per conoscere i parametri orbitali dobbiamo prima calcolare la risposta libera
del nostro sistema. Riscriviamo l'equazione $\ref{eq:2body_acc}$ senza
considerare le forze esterne
\begin{equation}
\dot{\bf v}(t)=-\frac{\mu_0}{r^3}{\bf r(t)}, v(0)=v_0
\end{equation}
La risposta libera descrive un'orbita che giace su un piano fisso, il piano
orbitale, definito dai vettori velocità e posizione. Si nota che l'accelerazione
è istante per istante opposta al vettore posizione, quindi se prendiamo in esame
il momento angolare per unità di massa {\bf h(t)}, definito come ${\bf
h(t)}={\bf r(t)}$ x ${\bf v(t)}$, la sua derivata è uguale a zero. Quindi ${\bf
h}=h{\bf e_h}$ è un vettore costante in direzione ${\bf e_h}$ (il polo dell'orbita) e
magnitudine h, che può essere presa come asse del sistema di riferimento.
Mediante il momento angolare possiamo anche definire il vettore eccentricità
\begin{equation}
{\bf v}(t) \times {\bf h} - \frac{\mu}{r(t)}{\bf r}(t)=\mu{\bf e}
\label{eq:eccentricity}
\end{equation}
$e=|{\bf e}|$ rappresenta  l'eccentricità dell'orbita e la sua direzione
l'origine della risposta libera. La forma dell'orbita dipende dalla sua
eccentricità, esistono tre tipi di coniche in funzione del valore di $e$
\begin{itemize}
  \item ellisse per $e<1$, che diventa una circonferenza quando e=0
  \item parabola per e=1 
  \item iperbole per $e>1$
\end{itemize}
Moltiplicando scalarmente per {\bf r(t)} l'equazione \ref{eq:eccentricity} si ottiene l'equazione
della risposta libera
\begin{equation}
h^2(t)-\mu r(t)=\mu r(t)e \cos{\theta(t)}
\label{eq:free_response}
\end{equation}
$\theta$, che rappresenta l'angolo tra $r$ ed $e$, è detto anomalia vera.
Riscrivendo l'equazione \ref{eq:free_response} in funzione del raggio orbitale,
otteniamo
\begin{equation}
r(t)=\frac{p}{1+e\cos{\theta{t}}} , p=\frac{h^2}{\mu}
\end{equation}
La costante p[m] è definita semilatus rectum. In figura è mostrata la geometria
del'orbita nel sistema di riferimento Local Vertical Local Horizontal (LVLH)
$\mathfrak{R_l}$=$\mathfrak{R}=\{C,\bar{l_1},\bar{l_2}=e_h,\bar{l_3}\}$ dove
$\bar{l_3}=\bar{e_r}=\bar{r}/r$ definisce la verticale locale, $l_1=e_{\theta}$
la orizzontale locale e $l_2=e_h$ è il polo orbitale.
Introduciamo il parametro $a$, che definisce il semi-asse maggiore dell'orbita
ellittica, e l'anomalia eccentrica $E$, il quale rappresenta l'angolo tra la
linea degli absidi e la linea tra $C_0$ , centro dell’ellisse, e $Q_1$ ,
definito come la proiezione del punto $P_1$ su un cerchio ausiliario di centro
$C_0$ come illustrato in figura.

