Consideriamo un sistema costituito da due corpi puntiformi $P_0$ e
$P_1$ aventi ripettivamente massa $m_0$ e $m_1$, definiamo la loro posizione e
velocità rispetto a un sistema inerziale $\mathfrak{R} \{
O,\bar{i_1},\bar{i_2},\bar{i_3}\}$, siano quindi $r_0$ e $r_1$ le posizioni e
$v_0$ e $v_1$ le velocità dei due corpi. Dalla seconda legge di Newton,
ricaviamo:
\begin{equation}
\begin{array}{l}
\dot{v_0}(t)=\frac{Gm_0m_1}{m_0r^3}r(t) + \frac{1}{m_0}F_0(t),v_0(0)=v_{00}\\\\
\dot{v_1}(t)=\frac{Gm_0m_1}{m_1r^3}r(t) + \frac{1}{m_1}F_1(t),v_1(0)=v_{10}
\end{array}
\end{equation}
dove: $r(t)=r_1(t)-r_0(t)$, $G=66.7x10^{-12} m^3kg^{-1}s^{-2}$ è la costante di
gravitazionale universale, mentre $F_0(t)$ e $F_1(t)$ sono due forze esterne
agenti sui corpi.
Studiamo il sistema nel suo insieme, quindi, sia $r_c$ il centro di massa del
sistema costituito dai due corpi, la sua accelerazione $\dot{v_c}$ risulta essere:
\[r_c=\frac{m_0}{m_0+m_1}r_0+\frac{m_1}{m_0+m_1}\]
\begin{IEEEeqnarray}{rCl}
\dot{v}(t)&=&\dot{v_1}(t)-\dot{v_0}(t)=\nonumber\\&=&-\frac{G(m_0+m_1)}{r^3}r(t)+\frac{1}{m_1}(F_1(t)-\frac{m_1}{m_0}F_0(t))\;,\;v(0)=v_0
\end{IEEEeqnarray}
\begin{equation}
\dot{v_c}(t)=\frac{F_1(t)+F_0(t)}{m_0+m_1},v_c(0)=v_{c0}
\end{equation}
Nel nostro caso il sistema è costituito dalla Terra e dal satellite in orbita
attorno ad essa. Poiché la massa della Terra è molto maggiore rispetto quella
del satellite, $m_0>>m_1$ le equazioni diventano:
\begin{equation}
\dot{v}(t)=-\frac{\mu_0}{r^3}r(t)+\frac{F_1(t)}{m_1}, v(0)=v_0
\label{eq:2body_acc}
\end{equation}
\begin{equation}
\dot{v_c}(t)=0, v_c(0)=v_{c0}
\end{equation}
dove $\mu_0=Gm_0$.
Il centro di massa $r_c$ lo si può approssimare al centro di massa della Terra
e, poiché la sua accelerazione è nulla, possiamo considera un sistema di
riferimento centrato in esso. L'equazione \ref{eq:2body_acc} riscritta
nel nuovo sistema di riferimento è nota come equazione ristretta del problema
dei due corpi.
Consideriamo un sistema costituito da due corpi puntiformi $P_0$ e
$P_1$ aventi ripettivamente massa $m_0$ e $m_1$, definiamo la loro posizione e
velocità rispetto a un sistema inerziale $\mathfrak{R} \{
O,\bar{i_1},\bar{i_2},\bar{i_3}\}$, siano quindi $r_0$ e $r_1$ le posizioni e
$v_0$ e $v_1$ le velocità dei due corpi. Dalla seconda legge di Newton,
ricaviamo:
\begin{equation}
\begin{array}{l}
\dot{\bf v_0}(t)=\frac{Gm_0m_1}{m_0r^3}{\bf r}(t) +
\frac{1}{m_0}{\bf F_0}(t),v_0(0)=v_{00}\\\\
\dot{\bf v_1}(t)=\frac{Gm_0m_1}{m_1r^3}{\bf r}(t) +
\frac{1}{m_1}{\bf F_1}(t),v_1(0)=v_{10}
\end{array}
\end{equation}
dove: $r(t)=r_1(t)-r_0(t)$, $G=66.7x10^{-12} m^3kg^{-1}s^{-2}$ è la costante di
gravitazionale universale, mentre $F_0(t)$ e $F_1(t)$ sono due forze esterne
agenti sui corpi.
Studiamo il sistema nel suo insieme, quindi, sia $r_c$ il centro di massa del
sistema costituito dai due corpi, la sua accelerazione $\dot{v_c}$ risulta essere:
\[r_c=\frac{m_0}{m_0+m_1}r_0+\frac{m_1}{m_0+m_1}r_1\]
\begin{equation}
\dot{\bf v}(t)=\dot{\bf v_1}(t)-\dot{\bf v_0}(t)=-\frac{G(m_0+m_1)}{r^3}{\bf
r}(t)+\frac{1}{m_1}({\bf F_1}(t)-\frac{m_1}{m_0}{\bf F_0}v), v(0)=v_0
\end{equation}
\begin{equation}
\dot{\bf v_c}(t)=\frac{{\bf F_1}(t)+{\bf F_0}(t)}{m_0+m_1},v_c(0)=v_{c0}
\end{equation}
Nel nostro caso il sistema è costituito dalla Terra e dal satellite in orbita
attorno ad essa. Poiché la massa della Terra è molto maggiore rispetto quella
del satellite, $m_0>>m_1$ le equazioni diventano:
\begin{equation}
\dot{\bf v}(t)=-\frac{\mu_0}{r^3}{\bf r}(t)+\frac{{\bf F_1}(t)}{m_1}, v(0)=v_0
\label{eq:2body_acc}
\end{equation}
\begin{equation}
\dot{\bf v_c}(t)=0, v_c(0)=v_{c0}
\end{equation}
dove $\mu_0=Gm_0$.
Il centro di massa $r_c$ lo si può approssimare al centro di massa della Terra
e, poiché la sua accelerazione è nulla, possiamo considera un sistema di
riferimento centrato in esso. L'equazione \ref{eq:2body_acc} riscritta
nel nuovo sistema di riferimento è nota come equazione ristretta del problema
dei due corpi.
